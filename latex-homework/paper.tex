\documentclass[letterpaper]{article}

% No page numbers.
\pagestyle{empty}

% Change the margins to 1 inch all around.
\usepackage[margin=1in]{geometry}

% for setting the linespace (\setstretch)
\usepackage{setspace}

% distance between the columns (for the multicols)
\setlength{\columnsep}{1cm}

% for compactitem
\usepackage{paralist}

% Hyperlinks inside the document.
\usepackage{hyperref}

\usepackage{verbatim}
\usepackage{xcolor}
\usepackage{booktabs}
\usepackage{longtable}
\usepackage{float}
\usepackage{graphicx}
\usepackage{listings}
\usepackage{amsmath}
\usepackage{tcolorbox}
\usepackage{amssymb}
\usepackage{parskip}
\usepackage{breqn}
\usepackage{booktabs}
\usepackage{pdfpages}
\usepackage{multirow}
\usepackage{parskip}
\usepackage{array}
\usepackage{amsthm}
\usepackage{bbm}
\usepackage{bbold}

% Bibliography management.
\usepackage[
  sorting=none,
  minbibnames=8,
  maxbibnames=9,
  block=space,
  backend=biber
]{biblatex}
\bibliography{bibliography}

% lorem ipsum generator.
\usepackage{lipsum}

%%%%% Extra math symbols %%%%%
\DeclareMathOperator*{\argmin}{arg\,min}
\DeclareMathOperator*{\argmax}{arg\,max}
\DeclareMathOperator*{\plim}{plim}

%%%%% Extra Commands %%%%%
\newcommand{\E}[1]{\mathbbm{E}\left[#1\right]}
\newcommand{\indicator}[1]{\boldsymbol{1}_{\{#1\}}}
\newcommand{\prob}[1]{\mathbbm{P}\left(#1\right)}
\newcommand{\mse}[1]{\text{MSE}\left(#1\right)}
\newcommand{\rmse}[1]{\text{RMSE}\left(#1\right)}
\newcommand{\var}[1]{\text{Var}\left(#1\right)}
\newcommand{\cov}[1]{\text{Cov}\left(#1\right)}
\newcommand{\corr}[1]{\text{Corr}\left(#1\right)}
\newcommand{\convP}{\xrightarrow{\mathbbm{P}}}
\newcommand{\Exp}[1]{\exp\left(#1\right)}
\renewcommand{\Vec}[1]{\boldsymbol{#1}}
\newcommand{\Mat}[1]{\mathbbm{#1}}
\newcommand{\tr}[1]{\text{tr}\left(#1\right)}


%%%%% Color definitions %%%%%
\definecolor{lightgray}{gray}{-1.9}
\definecolor{codegreen}{rgb}{0,0.6,0}
\definecolor{codegray}{rgb}{0.5,0.5,0.5}
\definecolor{codepurple}{rgb}{0.58,0,0.82}

%%%%%%%%%%%%%%%%%%%%%%%%%%%%%%%%%%%%%%%%%%%%%%%%%%%%%%%%%%%%%%%%%
\begin{document}

%%%%% Extra Environments %%%%%
\newtheorem*{theorem*}{Theorem}
\newtheorem*{claim*}{Claim}
\newtheorem*{lemma*}{Lemma}
\newtheorem*{def*}{Definition}

\newtheorem{theorem}{Theorem}
\newtheorem{claim}{Claim}
\newtheorem{lemma}{Lemma}
\newtheorem{Def}{Definition}

%%%%% Answer box environment %%%%%
\newtcolorbox{myanswerbox}[1][]{
  colback=white!, % Background color
  colframe=white!25!black, % Bourder color
  fonttitle=\bfseries, %  Font style of the title
  title=#1, %  Title
  arc=0mm, %  Rounded corners
  #1
}

%%%%% Python code environment %%%%%
% Style configuration for the code listings.
\lstdefinestyle{mystyle}{
    commentstyle=\color{codegreen},
    keywordstyle=\color{magenta},
    numberstyle=\tiny\color{codegray},
    stringstyle=\color{codepurple},
    basicstyle=\ttfamily\footnotesize,
    breakatwhitespace=false,         
    breaklines=true,                 
    captionpos=b,                    
    keepspaces=true,                 
    numbers=left,                    
    numbersep=5pt,                  
    showspaces=false,                
    showstringspaces=false,
    showtabs=false,                  
    tabsize=2
}

\lstset{style=mystyle}

%=============================
{
	\parindent0pt
	\setstretch{0.4}
	\ \\ \ \\ \ \\

	\hrulefill
	\vspace{0.0cm}
	\begin{spacing}{1.1}
	{	
		\flushleft
		\fontsize{22pt}{44pt}\selectfont 
		Title
	}\\
	\textsc{Course - Code}
	\end{spacing}

	\ \\ \ \\
	{
		\setstretch{0.2}
		\textbf{John Doe}\\
		Master in Arts\\
		University of Moria - Misty Mountains\par
	}
	\ \\

	\hrulefill
}

%\newpage
%=============================
\section*{Problem 1}
%%%%% PROBLEM STATEMENT %%%%%%%%%%%%%%%%%%%%%%%%%%%%%%%%%%%%%%%%%
\lipsum[1]
%%%%% QUESTION SEPARATOR %%%%%%%%%%%%%%%%%%%%%%%%%%%%%%%%%%%%%%%%
\begin{myanswerbox}[title={i}]
\lipsum[2]
\end{myanswerbox}
%%%%% Answer %%%%%

%\newpage
%=============================
%\section*{References}
% the \nocite command leads to the whole bibliography
% being displayed (without any \cite commands necessary).
% remove this command in order to get the "normal" behavior.
%\nocite{*}
%\newpage
%\printbibliography[heading=none]

\end{document}
